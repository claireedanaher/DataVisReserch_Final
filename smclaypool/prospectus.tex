\documentclass{proc}

\begin{document}

\title{%
    The Effects of Interaction on Bayesian Reasoning \\
    \large Type: Technique
}

\author{Saahil Claypool, Claire Danaher, and Alexander Shoop}

\maketitle

\section{Introduction}
Correctly interpreting inference has long been a challenging problem for humans regardless of education level. As data becomes increasingly available in diverse realms, a reliable system for effectively reasoning about Bayesian statistics becomes increasingly important. 

Researchers have developed a number of techniques for 
helping users to understand Bayesian models. However, effectiveness, generalizability, and scalability have varied. Prior techniques can be categorized into two (non-distinct) categories: 
frequency visualizations and structured textual representations \cite{Ottley2016}. Challenges associated with frequency diagrams include contradictory results when generalized to larger populations. Challenges associated with textual representations include that explaining a problem in a structured manner can be challenging.  

In this paper, we propose the use of interactive visualizations in conjunction with a question-style presentation of information to help end users make sense of Bayesian models. Past research has postulated that only specific portions of the problem space need to be known by the user to answer questions about the Bayesian models \cite{Gigerenzer1995}. Therefore, the cognitive load on the user can be reduced by constraining the information presented using interaction and probing questions. 

While prior work has been completed in interactive Bayesian representations, visualizations were not constrained and still forced the user to keep a larger portion of the Bayesian model in memory. Our visualizations aim to fix this by showing only pertinent information to the users' queries. 

\section{One-sentence description}

We propose exploring the use of interaction to reduce cognitive load through user interactions whereby making it easier to understand visual representation of Bayesian probability models thus improving a users ability to make accurate inferences.

 

\section{Project Type}

This will be a technique paper. The goal would be to create a new technique and evaluate how effective it is for Bayesian Inference tasks. 

\section{Audience} 

The audience for this paper is researchers interested in understand how to better present Bayesian probabilities. This paper may be of interest to individuals interested in understanding how the selective presentation of information can help improve task based activities. Without this work, there will continue to be a large number of false assumptions caused by a misunderstanding of Bayesian probabilities. 

\section{Approach}
\subsection{Details}
\begin{enumerate}
\item Create a small interactive tool

The 'tool' will be provided as a reference similar to prior studies. This will hopefully help the user reason
about the problem by providing a visual aid. So, while the numbers will be different, it should allow them to 
make a similar 'query' and thus answer their question about their sample. 

\item Quiz the user

Given the interactive template or tool, the user will be asked some Bayesian question about a new sample. 
This will be compared to a set of users without the interactive tool. 

The question will asked similar to prior work: 
\begin{quote}
    If a woman has breast cancer, the
    probability is 80\% that she will get a positive mammography. If a woman
    does not have breast cancer, the probability is 9.6\% that she will also
    get a positive mammography. A woman in this age group had a positive
    mammography in a routine screening \cite{Gigerenzer1995}.
\end{quote}
In the base scenario for this proposal, we will conduct a small number of user studies. The results of this study will serve as proof of feasibility for a larger study that could produce statistically significant results. However, the larger study is likely outside the scope of this paper.
\end{enumerate}


\subsection{Evidence for Success}

User study will indicates that people do not need to keep the entire Bayesian model in their head to make inferences, but 
rather just the few pertinent numbers for their estimations \cite{Gigerenzer1995}. So, by allowing people
to narrow down the Bayesian model to answer just the question asked, maybe they will be able to ignore the other
distracting numbers and better answer the question. 

This interactive design has been tested before \cite{Tsai2011}, but failed to reach significant results.
This could be because the visualizations provided did not hide extraneous information, nor did it use aligned axes
for the comparison of categories. Both of these together may have made it harder for users to reason about the problem. 

Further, with modern technologies like JavaScript, we should be able to create a richer interactive model 
that is more natural for the user to use than an excel sheet and visual basic scripts. 


\section{Best-case Impact Statement}
This paper will be successful if it is able to provide definitive results on an interactive design helping in 
Bayesian reasoning. 

\section{Major Milestones}

\begin{enumerate}
    \item Create the interactive tool
    \item Create a testing framework to run users through

    including the data and questions they will be asked about the data

    \item Run a small set of pilot users

    Tweak study as necessary

    \item Run a larger set of users

    \item OR run a case study detailing how a smaller set of users worked with the interactive representation

    This will be a backup study if we cannot feasibly run a user study with enough participants. 
    
\end{enumerate}

\section{Obstacles}

\subsection{Major obstacles} % (if these fail, the project is over)

\begin{enumerate}
    \item Cannot get users within the next 5 weeks
    \item Users cannot figure out how to interact with a tool on a website
    \item There isn't a guarantee that non-technical users will intuitively click around on a graph. 

\end{enumerate}

\subsection{Minor obstacles}

\begin{enumerate}
    \item Scaffolding the experiment is slightly undefined
\end{enumerate}

\section{Resources Needed}
\begin{enumerate}
    \item Willing participants for the case study
    \item *Potentially* Mechanical turk survey users
\end{enumerate}


\section{5 Related Publications}
\begin{enumerate}
    \item How to improve Bayesian reasoning \cite{Gigerenzer1995} \item
    Assessing the effect of visualizations on Bayesian reasoning through
    crowdsourcing \cite{Micallef2012} 

    \item Improving Bayesian Reasoning: The
    Effects of Phrasing, Visualization, and Spatial Ability \cite{Ottley2016}
    \item Interactive visualizations to improve Bayesian reasoning
    \cite{Tsai2011} 

    \item Pictorial representations in statistical reasoning
    \cite{Brase2009} 

    \item Using tree diagrams without numerical values in
    addition to relative numbers improves students' numeracy skills: A
    randomized study in medical education \cite{Friederichs2014}

\end{enumerate}

\section{Define Success}

Success for this project will be achieved through designing an interactive tool that leads the user through a set of steps required to interpret Bayesian statistics. 

We will conduct a case studies of a small sample of users and have a discussion of how they ended up 
actually interacting (or not) with the interactive model. This could provide insight on how untrained users are to interact with a computer to solve problems. These case study would provide a lower barrier for 
a useful outcome of our paper. 

If time allows, this optimized visualization from the case study will be evaluated using mechanical turk surveys. This would allow us to test the hypothesis that groups that answer the questions using the interactive model will out-perform those with just text. 

\bibliographystyle{abbrv}
\bibliography{prospectus}
\end{document}
