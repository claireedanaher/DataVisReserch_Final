\documentclass{proc}

\begin{document}

\title{%
    The Effects of Interaction on Bayesian Reasoning \\
    \large Type: Technique
}

\author{Saahil Claypool}

\maketitle

\section{Introduction}

Inference has long been a hard problem for humans, 
regardless of education. But, as evidence based reasoning becomes more
important, a flexible system for reasoning about Bayesian statistics is 
necessary. 

While researchers have developed a large number of different techniques to 
help users understand Bayesian models, the techniques vary in both effectiveness,
generalizability, or scalability. Prior effective techniques can be categorized into two (non-distinct) categories: 
frequency visualizations and structured textual representations \cite{Ottley2016}. But, frequency diagrams
have had contradictory results when generalized to larger populations, and textual representations
may be more problem-specific. It may be harder to write a paragraph explaining a problem in a structured manner, whereas
graphs can be recreated from the data without very little effort. 

In this paper, we discuss the use of interactive visualizations and a question-style
presentation of information to help end users make sense of Bayesian models. We postulate that, 
because only specific portions of the problem space need to be known by the user to answer 
questions about the Bayesian models \cite{Gigerenzer1995}, we can reduce the cognitive load on the user by 
constraining the information presented to the user with interaction and probing questions. While prior work has been 
in interactive Bayesian representations, those visualizations do not constrain the problem, and still
force the user to keep a larger portion of the Bayesian model in memory. Our visualizations aim to fix this by 
showing only pertinent information to the users' queries. 



\section{One-sentence description}

To create an easier to understand representation for Bayesian probability models, we explore the user
of interaction to reduce the information presented to a user, and thus improve their ability to make accurate
inferences. 

\section{Project Type}

Technique 

The goal would be to create a new technique and evaluate how effective it is for Bayesian Inference tasks. 

\section{Audience} 

The outcomes of this project will help researchers understand how to better present Bayesian probabilities, 
and also understand how the selective presentation of information can help improve task based activities. 

Without this work, there will continue to be a large number of false assumptions caused by a misunderstanding of 
Bayesian probabilities. 

\section{Approach}
\subsection{Details}
\begin{enumerate}
\item Create a small interactive tool

The 'tool' will be provided as a reference similar to prior studies. This will hopefully help the user reason
about the problem by providing a visual aid. So, while the numbers will be different, it should allow them to 
make a similar 'query' and thus answer their question about their sample. 

\item Quiz the user

Given the interactive template or tool, the user will be asked some Bayesian question about a new sample. 
This will be compared to a set of users without the interactive tool. 

The question will asked similar to prior work: 
\begin{quote}
    If a woman has breast cancer, the
    probability is 80\% that she will get a positive mammography. If a woman
    does not have breast cancer, the probability is 9.6\% that she will also
    get a positive mammography. A woman in this age group had a positive
    mammography in a routine screening \cite{Gigerenzer1995}.
\end{quote}
\end{enumerate}


\subsection{Evidence for Success}

indicates that people do not need to keep the entire Bayesian model in their head to make inferences, but 
rather just the few pertinent numbers for their estimations \cite{Gigerenzer1995}. So, by allowing people
to narrow down the Bayesian model to answer just the question asked, maybe they will be able to ignore the other
distracting numbers and better answer the question. 

This interactive design has been tested before \cite{Tsai2011}, but failed to reach significant results.
This could be because the visualizations provided did not hide extraneous information, nor did it use aligned axes
for the comparison of categories. Both of these together may have made it harder for users to reason about the problem. 

Further, with modern technologies like JavaScript, we should be able to create a richer interactive model 
that is more natural for the user to use than an excel sheet and visual basic scripts. 


\section{Best-case Impact Statement}
This paper will be successful if it is able to provide definitive results on an interactive design helping in 
Bayesian reasoning. 

\section{Major Milestones}

\begin{enumerate}
    \item Create the interactive tool
    \item Create a testing framework to run users through

    including the data and questions they will be asked about the data

    \item Run a small set of pilot users

    Tweak study as necessary

    \item Run a larger set of users

    \item OR run a case study detailing how a smaller set of users worked with the interactive representation

    This will be a backup study if we cannot feasibly run a user study with enough participants. 
    
\end{enumerate}

\section{Obstacles}

\subsection{Major obstacles} % (if these fail, the project is over)

\begin{enumerate}
    \item Cannot get users within the next 7 weeks
    \item Users cannot figure out how to interact with a tool on a website

    There isn't a guarantee that non-technical users will intuitively click around on a graph. 

\end{enumerate}

\subsection{Minor obstacles}

\begin{enumerate}
    \item Scaffolding the experiment is slightly undefined
\end{enumerate}

\section{Resources Needed}
\begin{enumerate}
    \item A lab to run users
    \item OR Mechanical Turk
\end{enumerate}


\section{5 Related Publications}
\begin{enumerate}
    \item How to improve Bayesian reasoning \cite{Gigerenzer1995} \item
    Assessing the effect of visualizations on Bayesian reasoning through
    crowdsourcing \cite{Micallef2012} 

    \item Improving Bayesian Reasoning: The
    Effects of Phrasing, Visualization, and Spatial Ability \cite{Ottley2016}
    \item Interactive visualizations to improve Bayesian reasoning
    \cite{Tsai2011} 

    \item Pictorial representations in statistical reasoning
    \cite{Brase2009} 

    \item Using tree diagrams without numerical values in
    addition to relative numbers improves students' numeracy skills: A
    randomized study in medical education \cite{Friederichs2014}

\end{enumerate}

\section{Define Success}

Optimally, we will have (statistically) significant results indicating groups with an interactive
model will out-perform those with just text. 

We may be able to make do with case studies of smaller users and a discussion of how they ended up 
actually interacting (or not) with the interactive model. This could provide insight on how untrained
users are to interact with a computer to solve problems. These case studies would provide a lower barrier for 
a useful outcome of our paper. 

\bibliographystyle{abbrv}
\bibliography{prospectus}
\end{document}
